\section{Usage Instructions}
\begin{description}
    \item[Compilation]
	Adjust the path to the kernel source in your \texttt{Makefile}.
	Then, use the \texttt{make} command to compile the C modules required by the python package.
    \item[Kernel Preparation]
        A kernel image with debug symbols must be compiled or obtained.
        Afterwards \texttt{objdump} can be used to extract the debug-symbols in a textual format readable by the tool.
        Pay attention to use a recent version of objdump, as older ones might fail to read debug-symbols.
        Next, the type\_parser script is used to create the initial cleaned-up data model.
        As noted earlier this last step may take some time.
        It will save its results in the data.dumpc file, which is later used to quick load the data model.
        \begin{lstlisting}[frame=single,caption=Initialisation Workflow,label=lst:workflow]
 # objdump -g vmlinux > vmlinux.dbgdump
 # python type_parser.py init vmlinux.dbgdump
        \end{lstlisting}
    \item[Additional Information Acquirement]
	Certain information is not available in the kernel’s debug symbols.
	Linked lists have been implemented and require an additional setup step to work correctly:
        \begin{lstlisting}[frame=single,caption=Linked List Initialisation Workflow,label=lst:workflow]
 # python type_relater.py /path/to/compiled/kernel/source meta_info.dump > /dev/null
        \end{lstlisting}
    \item[Memory Unlocking]
        In order to access the \texttt{/dev/mem} device without any restrictions, the supplied memunlocker kernel module needs to 		be loaded: \begin{lstlisting}[frame=single,caption=Unblocking /dev/mem,label=lst:unlocking]
 # sudo insmod memunlocker/memunlocker.ko
         \end{lstlisting}
    \item[Using the Shell]
	Interactive kernel symbol exploration can be done via the python shell. First of all the precalculated debug symbol data 		has to be loaded. Then a memory image (can be the \texttt{/dev/mem} device or a physical memory dump of a virtual machine) 		is mapped using the memory.c python plugin.
	\begin{lstlisting}[frame=single,caption=Starting up interactive shell,label=lst:shellstartup]
 # python
 >>> from tools import *
 >>> types, names, addresses = init(memory_image, parents=True, linked_lists=True)
 >>> it = kernel_name('init_task')
 >>> init = it.children.next
 >>> init
 <Memory <c_types.Struct instance 'task_struct'> @0xffff88001ccc8000>
 >>> init.pid
 <Memory <c_types.Member instance 'pid'> @0xffff88001ccc8268>
 >>> print init.pid
 1
 >>> 
	\end{lstlisting}	

    \item[Symbol Comparision]
	In order to be able to compare a symbol at different time snapshots, a second memory dump has to be mapped. 
	Then the location of the level 4 page table directory has to be published to the memory python module.
	\begin{lstlisting}[frame=single,caption=Mapping a second memory image,label=lst:map]
 >>> memory.map(second_image, filesize, mapsize (usual same as filesize), 1)

 >>> pgt = kernel_name('__ksymtab_init_level4_pgt')
 >>> memory.set_init_level4_pgt(int(pgt.value.get_value()))
	\end{lstlisting}
	If everything is set up symbols can be compared by first getting a reference with \texttt{ref = kernel\_name('symbol')} and 		then using the \texttt{memory\_manager's} \texttt{.memcmp()} functionality: \texttt{ref.memcmp()}.
	For a complete example see \texttt{diff.py}.
    \item[Report generation]
	Several components are capable of generating (more or less) detailed reports on actions and failures.
	This information will be of great use for improving the tool and debugging errors it might have made.
        \begin{lstlisting}[frame=single,caption=Generating Reports,label=lst:workflow]
 # python troubles.py           > reports/trouble-areas.xml
 # python type_relater.py /path/to/kernel/source meta_info.dump \
                                > reports/linked-lists-parsing.xml
 # python tools/linked_lists.py > reports/linked-lists-autoassociate.xml
        \end{lstlisting}
\end{description}
